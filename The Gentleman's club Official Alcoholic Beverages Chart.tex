\documentclass[12pt,a4paper,oneside,norsk]{article} 
\usepackage[utf8]{inputenc}
%\usepackage[norsk]{babel}
\usepackage{amsmath}
\usepackage{amsfonts}
\usepackage{amssymb}
\usepackage{makeidx}
\usepackage{graphicx}
\usepackage{hyperref}
\usepackage[left=2cm,right=2cm,top=2cm,bottom=2cm]{geometry}
\usepackage{float}
\usepackage{multirow}
\usepackage{verbatim} %for å kommentere ut ting
\usepackage[nottoc,numbib]{tocbibind}
\usepackage{chngpage} % allows for temporary adjustment of side margins
\usepackage[parfill]{parskip} %for avsnitt
\usepackage[yyyymmdd,hhmmss]{datetime}
\usepackage{comment} 
\usepackage{caption}
\captionsetup[figure]{labelformat=empty}

\raggedbottom

\usepackage{makeidx}
\makeindex

\begin{document}
%her kommer forsiden:
    \begin{titlepage}
    \begin{center}
    \ \\
    \ \\
    \ \\
    The Gentleman's Club \\
    \ \\
    \ \\
    \ \\
    \ \\
    \ \\
    \ \\{\large \bfseries
    The Gentleman's Club Official Alcoholic Beverages Chart\\
    }
    \ \\
    \ \\
\begin{figure} [H]
\centering
\includegraphics[scale=0.45]{Bilder/forside.jpg} %scale justerer størrelsen på bilde. Linjen etterpå er mappen bilde ligger i.
\end{figure}
    \ \\
    \ \\
        {\large
    Alcoholic Beverages Chart\\
    }
    \ \\
    {\today\ \\}
    \ \\
    \end{center}
    \end{titlepage}
    
    \thispagestyle{empty}
\newpage

\setcounter{page}{1}
\pagenumbering{arabic}
%her er innholdsfortegnelsen. Den lages automagisk
\tableofcontents
\newpage

%HER ER EN LITEN BRUKSANVISNING
%Nedenfor er en mal til hvordan man lager en subsubsection
\begin{comment}
%start å klipp og lim herifra, og lim det inn under riktig "subsection":

\subsubsection{Bryggeri: NAVN PÅ ØL}
\paragraph{Kommentar:} SKRIV DIN MENING HER
\newline
-- -- ( SKRIV NAVN OG DATO)

\begin{figure} [H] %[H] hindrer latex i å bestemme hvor bilde skal stå. men det kommer der du vil ha det.
\centering
\includegraphics[scale=0.60, angle=0]{Bilder/Øl/outcomeinterruption.png} %scale justerer størrelsen på bilde, angle rotasjonen. Linjen etterpå er mappen bilde ligger i.
\caption{SKRIV EN BILDE TEKST.}
\end{figure}
\newpage
%stopp med klipp og lim her! --------------------------
\end{comment}


\section{Øl}
\subsection{Bayer}

\subsubsection{Det Lille Bryggeriet: Bjønnøl}
\paragraph{Kommentar:}Kjedelig og smakløst øl. Lever ikke opp til navnet og er ikke verdt pengene.
\newline
-- -- Isak 13.04.2014

\begin{figure} [H]
\centering
\includegraphics[scale=1.00]{Bilder/Ol/bjonnol.png} %scale justerer størrelsen på bilde. Linjen etterpå er mappen bilde ligger i.
\caption{Bjønnøl fra "Det Lille Bryggeri"}
\end{figure}


\newpage
\subsection{Pils}

\subsubsection{Det Lille Bryggeriet: Birkebeinerpils}
\paragraph{Kommentar:}Nok en skuffende øl fra Det Lille Bryggeriet. Kjedelig og en litt ubehagelig smak i svelget. 
\newline
-- -- Isak 13.04.2014

\begin{figure} [H]
\centering
\includegraphics[scale=0.1, angle=270]{Bilder/Ol/Birkebeiner.jpg}
\caption{Birkebeinerpils fra "Det Lille Bryggeriet"}
\end{figure}

\newpage
\subsubsection{E.C. Dahls Bryggeri: Dahls Pils}
\paragraph{Kommentar:}En ok industripils til prisen man betaler for den. Den har et preg av banan og fargen er som morgen-piss. Nytes best full, kan like gjerne drikkes rett fra flasken som fra glass. Da sparer man litt oppvask. Selv om ølen har et kipt industripreg, er den vist nok gunstig for bartevekst. Bivirkning ved ølen er utydelig tale og trang til å skyte inn med ordet "sjø" stadig vekk.
\newline
-- -- Anders 13.04.2014

\begin{figure} [H]
\centering
\includegraphics[scale=0.1, angle=0]{Bilder/Ol/dahls.jpg}
\caption{Dahls Pils fra "E.C. Dahls Bryggeri"}
\end{figure}

\newpage
\subsection{Hveteøl}
\subsubsection{Ægir Bryggeri: Witbier}
\paragraph{Kommentar:} Til å være hveteøl var denne mindre søt enn jeg hadde forventet. Den var lett og lys, og egner seg kanskje i mer sommerlige situasjoner enn jeg drakk den i. Selv savnet jeg litt mer ramm smak, men det skal man kanskje ikke forvente av en hveteøl. Hadde jeg hatt valget mellom denne og en Dahls ville jeg tatt denne, så lenge jeg ikke måtte betale for den selv. For lite smak for pengene rett å slett.
\newline
-- -- Anders 14.04.2014

\begin{figure} [H]
\centering
\includegraphics[scale=0.1, angle=0]{Bilder/Ol/EgirBryggeriWitbier.jpg}
\caption{Witbier fra "Ægir Bryggeri"}
\end{figure}

\newpage
\section{Vin}

\newpage
\section{Sprit}
\subsection{Whisky}
\subsubsection{Lagavulin: Island Single Malt Scotch Whisky 16 Years}

\paragraph{Kommentar:}Sterk, men behagelig røkpreg. En "varmende" avrunding, men ikke en typisk rund whisky. Til en middels høy pris på polet er dette verdt hver krone. Et må ha i barskapet. Plix ikke bland cola i dette a.
\newline
-- -- Anders 13.04.2014

\begin{figure} [H]
\centering
\includegraphics[scale=0.1, angle=0]{Bilder/Sprit/Lagavulin16aar.jpg}
\caption{Island Single Malt Scotch Whisky 16 Years fra "Lagavulin"}
\end{figure}


\newpage
\subsection{Cognac}
\end{document}